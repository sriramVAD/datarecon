\documentclass{article}
\usepackage{fixltx2e}
\title{Improper Integrals}
\author{Sriram Vadlamani}
\begin{document}
\maketitle
\newpage
\tableofcontents
\newpage
\section{Definition}
When a function is not defind for either or both of the bounds in an integral, it is known as an improper integral.\\
$\int _ { a } ^ { b } f ( t ) d t$\\
$f: [ a , b [ \rightarrow R $\\
\textit{f} continous on $[ a , b [$\\

E.g:\\
$\int _ { 1 } ^ { + \infty } e ^ { - t } d t$\\
\section{Convergence and Divergence}
let $f: [a, b[ \rightarrow R $  and \textit{f} be continous on $[a, b[$.\\
we say that the integral is convergent if\\
$$\lim_{x\to b} \int_{a}^{x} f(t) dt$$ is a finite value.\\
else, it is divergent.\\
\subsection{Proposition 1}
let \textit{f} and \textit{g} be two functions on $[ a, b [$ such that, both integrals 
$\int_{a}^{b} f(t) dt$ and $\int_{a}^{b} g(t) dt$ converge.\\
Then
$ \forall \alpha \in \mathbb{R} $, $\int_{a}^{b} (\alpha f(t) + g(t))$ converges and linearity can be used to split the two integrals.\\
\subsection{Riemann Functions}
Theorem:\\
\begin{itemize}
    \item $\int_{0}^{1} dt / t ^ \alpha $ converges $ \Leftrightarrow $ $ \alpha < 1 $\\
    \item $\int_{1}^{+\infty} dt / t ^ \alpha $ converges $ \Leftrightarrow $ $\alpha > 1$\\
    \item $\int_{0}^{+\infty} dt / t ^ \alpha $ is always divergent.\\
\end{itemize}
\subsection{Positive funtions improper integrals}
let $f: [ a, b [ \rightarrow $ R\textsubscript{+} and $g: [ a, b [ \rightarrow $ R\textsubscript{+}, continous on $ [a, b [ $.\\
if $f(t) < g(t)$, then\\
\begin{itemize}
    \item if $\int _ { a } ^ { b } g ( t ) d t$ is conv \Rightarrow $\int _ { a } ^ { b } f ( t ) d t$ is conv.\\ 
    \item if $\int _ { a } ^ { b } f ( t ) d t$ is divergent \Rightarrow $\int _ { a } ^ { b } g ( t ) d t$ is divergent.\\
    \item if $f(t) =  o (g(t)) $ then if $\int _ { a } ^ { b } g ( t ) d t$ is conv. \Rightarrow $\int _ { a } ^ { b } f ( t ) d t$ is conv.\\
    \item if $f(t) \sim g(t) $ then $\int _ { a } ^ { b } f ( t ) d t$ and $\int _ { a } ^ { b } g ( t ) d t $ have the same nature\\ 
\end{itemize}
\section{Cheat sheet}
\subsection{Different Integration methods}
\subsection{Integration by parts}
When the integral is in the form of $f(t)\cdot g(t)$ we can do the integration by splitting them into the following:\\
$\int_{a}^{b} u \cdot v'$ = $u\cdot v + $ $\int_{a}^{b} v\cdot u'$\\
\end{document}
