\documentclass{article}
% Options for packages loaded elsewhere
\PassOptionsToPackage{unicode}{hyperref}
\PassOptionsToPackage{hyphens}{url}
%
\documentclass[
]{article}
\usepackage{lmodern}
\usepackage{amssymb,amsmath}
\usepackage{ifxetex,ifluatex}
\ifnum 0\ifxetex 1\fi\ifluatex 1\fi=0 % if pdftex
  \usepackage[T1]{fontenc}
  \usepackage[utf8]{inputenc}
  \usepackage{textcomp} % provide euro and other symbols
\else % if luatex or xetex
  \usepackage{unicode-math}
  \defaultfontfeatures{Scale=MatchLowercase}
  \defaultfontfeatures[\rmfamily]{Ligatures=TeX,Scale=1}
\fi
% Use upquote if available, for straight quotes in verbatim environments
\IfFileExists{upquote.sty}{\usepackage{upquote}}{}
\IfFileExists{microtype.sty}{% use microtype if available
  \usepackage[]{microtype}
  \UseMicrotypeSet[protrusion]{basicmath} % disable protrusion for tt fonts
}{}
\makeatletter
\@ifundefined{KOMAClassName}{% if non-KOMA class
  \IfFileExists{parskip.sty}{%
    \usepackage{parskip}
  }{% else
    \setlength{\parindent}{0pt}
    \setlength{\parskip}{6pt plus 2pt minus 1pt}}
}{% if KOMA class
  \KOMAoptions{parskip=half}}
\makeatother
\usepackage{xcolor}
\IfFileExists{xurl.sty}{\usepackage{xurl}}{} % add URL line breaks if available
\IfFileExists{bookmark.sty}{\usepackage{bookmark}}{\usepackage{hyperref}}
\hypersetup{
  hidelinks,
  pdfcreator={LaTeX via pandoc}}
\urlstyle{same} % disable monospaced font for URLs
\usepackage{listings}
\newcommand{\passthrough}[1]{#1}
\lstset{defaultdialect=[5.3]Lua}
\lstset{defaultdialect=[x86masm]Assembler}
\usepackage{graphicx}
\makeatletter
\def\maxwidth{\ifdim\Gin@nat@width>\linewidth\linewidth\else\Gin@nat@width\fi}
\def\maxheight{\ifdim\Gin@nat@height>\textheight\textheight\else\Gin@nat@height\fi}
\makeatother
% Scale images if necessary, so that they will not overflow the page
% margins by default, and it is still possible to overwrite the defaults
% using explicit options in \includegraphics[width, height, ...]{}
\setkeys{Gin}{width=\maxwidth,height=\maxheight,keepaspectratio}
% Set default figure placement to htbp
\makeatletter
\def\fps@figure{htbp}
\makeatother
\setlength{\emergencystretch}{3em} % prevent overfull lines
\providecommand{\tightlist}{%
  \setlength{\itemsep}{0pt}\setlength{\parskip}{0pt}}
\setcounter{secnumdepth}{-\maxdimen} % remove section numbering

\author{}
\date{}

\usepackage{amsmath}
\usepackage{amssymb}
\usepackage{standalone}
\title{Discrete Fourier Transform and Fast Fourier Transforms}
\author{Sriram Vadlamani}
\begin{document}
\maketitle
\newpage
\section{The fourier series}
The Discrete Fourier Transforms acronymed as DFT can be derived from the formula of the fourier series.\\
Let's have a look at the general formula of the Fourier Series:\\
\\
$f(x)$ = $a_{0}$ + $\displaystyle \sum_{n = 1}^{+ \infty}(a_{n}\cos(\frac{n \pi x} {L}) + b_{n}\sin(\frac{n \pi x}{L}))$\\
\\
\subsection{Arriving at the Fourier series from the DFT}
We will try to prove that the DFT are just the same as the Fourier series general formula. We can start by looking at the general DFT formula\\
$\forall f_{k} $ where $k \in \mathbb{N}$ are vectors and $\hat{f}_{k} $ are the frequencies of the each $f_{k}$ vectors i.e, the fourier transform.\\
\\
$\hat{f}_{k}$ = $\displaystyle \sum_{j = 0}^{n - 1} f_{j} e^{-\iota 2 \pi j k / n}$\\
\\
From here, let's use $e^{\iota \theta} = \cos(\theta) + \iota \sin(\theta)$\\
\\
Then, we have:\\
\\
$\hat{f}_{k} $ = $\displaystyle \sum_{j = 0}^{n - 1} \{ (f_{j} \cos(\frac{2 \pi j k} {n})) - (f_{j} \iota \sin(\frac{2 \pi j k}{n})) \} $\\
\\
For this proof, let's assume $k = 1$\\
\\
For $j = 0$ the cosine term equals 1 and the sine term nullifies and hence becomes $f_{0}$\\
\\
simplifying on the same basis, we get:\\
\\
$f_{0}$ + $ f_{1}(\cos(\frac{2 \pi}{n}) - \iota \sin(\frac{2 \pi}{n})) $ + $f_{2} (\cos(\frac{4 \pi}{n}) - \iota \sin(\frac{4 \pi}{n}))$ + $ \cdots $
 + $f_{n}(\cos(\frac{2 (n - 1) \pi }{n}) - \iota \sin(\frac{2 (n - 1) \pi }{n}))$\\
 \\
 Now, factoring the cosines and sines i.e, rearranging them, we get:\\
 \\
 $\hat{f}_{1}$ = $f_{0}$ + $\displaystyle \sum_{j = 1}^{n} f_{n}(\cos(\frac{2 \pi j}{n}) - \iota \sin(\frac{2 \pi j}{n}))$\\
 \\
 We can see a clear relation with the general fourier expression with $a_{0} = f_{0}$, $a_{n} = f_{n}$ and $b_{n} = - \iota f_{n}$\\
 \\
\section{DFT matrix}
From the general formula of the DFT $\hat{f}_{k}$ = $\displaystyle \sum_{j = 0}^{n - 1} f_{j} e^{-\iota 2 \pi j k / n}$ we can say that\\
\\
$\hat{f}_{k}$ are multiples of $e^{\frac{- 2 \pi \iota }{n}}$\\
\\
So let's say $\omega_{n}$ = $e^{\frac{- 2 \pi \iota }{n}}$\\
\\
let the vector of $f_{k}$ be\\
$$\begin{bmatrix} f_{1}\\ f_{2} \\ \vdots \\ f_{k} \end{bmatrix}$$
and let the $\hat{f}_{k}$ vector be\\
$$\begin{bmatrix} \hat{f}_{1}\\ \hat_{f}_{2}\\ \vdots \\ \hat{f}_{k} \end{bmatrix}$$
\\
and the DFT matrix would hence be:\\
\\
$$ 
\begin{bmatrix}
1 & 1 & 1 & \cdots & 1 \\
1 & \omega_{n} & \omega_{n} & \cdots & \omega_{n}^{n-1} \\
1 & \omega_{n} & \omega_{n}^{2} & \cdots & \omega_{n}^{2 (n - 1)}\\
1 & \vdots & \vdots & \vdots & \vdots\\
1 & \omega_{n}^{n-1} & \omega_{n}^{2 (n - 1)}& \cdots & \omega_{n}^{(n-1)^{2}}
\end{bmatrix}
$$
\\
Therefore we can say that:\\
$\hat{f}_{k} = M \cdot f_{k}$\\
\\
The above matrix multiplication gives us the $\hat{f}_{k}$ vectors which are nothing but the frequency of the of each of the $f_{k}$ vectors.\\
\\
\section{Fast Fourier Transforms (FFT)}
The FFT is an algorithm that performs the DFT in a more efficient way. Let's assume the number of vectors $f_{k}$ are a clean power of two.
let k = 1024.\\
in that case the DFT can be written as:\\
\\
$\hat{f}_{k} = M_{1024} \cdot f_{k}$
we can split the $M_{1024}$ matrix as the following:\\
\\
$M_{1024}$ = $\begin{bmatrix} I_{512} & D_{512}\\ I_{512} & -D_{512} \end{bmatrix}$ $\cdot$ $\begin{bmatrix} F_{512} & 0\\ 0 & F_{512} \end{bmatrix}$ $\cdot$ $\begin{bmatrix} f_{even}\\ f_{odd} \end{bmatrix}$\\
\\
Where I is the identity matrix and D is the diagonal matrix of the given order.\\
\\
Now, the $F_{512}$ matrix can be further split into 2 $F_{256}$ and 4 $F_{128}$ and so on until we reach a 2 x 2 matrix, whose value can be stored to compute the other values, this is also known as dynamic programming.\\
\section{A simple noise reduction program in python}

\begin{lstlisting}[language=Python]
# Import statements
import os
import sys
import numpy as np
import matplotlib.pyplot as plt
\end{lstlisting}

\begin{lstlisting}[language=Python]
plt.rcParams['figure.figsize'] = [16, 12]
plt.rcParams.update({'font.size': 18})
\end{lstlisting}

\begin{lstlisting}[language=Python]
# create a simple signal with two frequencies
dt = 0.001
t = np.arange(0,1,dt)
f = np.sin(2*np.pi*50*t) + np.sin(2*np.pi*120*t) # some random frequencies
f_clean = f
f = f + 2.5*np.random.randn(len(t))

plt.plot(t,f,color='orange',LineWidth=1.5,label='Noisy')
plt.plot(t,f_clean,color='black',LineWidth=2,label='Clean')
plt.xlim(t[0], t[-1])
plt.legend()
\end{lstlisting}

\begin{lstlisting}
<matplotlib.legend.Legend at 0x7f17636c9d60>
\end{lstlisting}

\begin{figure}
\centering
\includegraphics{FFT_files/FFT_2_1.png}
\caption{png}
\end{figure}

\begin{lstlisting}[language=Python]
n = len(t)
fhat = np.fft.fft(f,n)  # It's that simple in numpy!
PSD = fhat * np.conj(fhat) / n # PSD = Power spectral density
freq = (1/(dt*n)) * np.arange(n)
L = np.arange(1,np.floor(n/2),dtype='int')

fig,axs = plt.subplots(2,1)
plt.sca(axs[0])
plt.plot(t,f,color='c',LineWidth='1.5',label='Noisy')
plt.plot(t,f_clean,color='k',LineWidth='1.5',label='clean')
plt.xlim(t[0],t[-1])
plt.legend()

plt.sca(axs[1])
plt.plot(freq[L],PSD[L],color='c',LineWidth='1.5',label='noisy')
plt.xlim(freq[L[0]], freq[L[-1]])
plt.xlabel('frequency')
plt.ylabel('PSD')
plt.legend()

plt.show()
\end{lstlisting}

\begin{lstlisting}
/usr/lib/python3.8/site-packages/numpy/core/_asarray.py:85: ComplexWarning: Casting complex values to real discards the imaginary part
  return array(a, dtype, copy=False, order=order)
\end{lstlisting}

\begin{figure}
\centering
\includegraphics{FFT_files/FFT_3_1.png}
\caption{png}
\end{figure}

\begin{lstlisting}[language=Python]
'''
We can clearly see two spikes in the above graph that are 50hz and 120hz, which is no surprise as they are the
frequencies that we constructed. So we can choose a number, say 100 and any Fourier coeff. below 100 can be zeroed out 
to give us pure frequencies of 50 and 120hz.
'''
\end{lstlisting}

\begin{lstlisting}
'\nWe can clearly see two spikes in the above graph that are 50hz and 120hz, which is no surprise as they are the\nfrequencies that we constructed. So we can choose a number, say 100 and any Fourier coeff. below 100 can be zeroed out \nto give us pure frequencies of 50 and 120hz.\n'
\end{lstlisting}

\begin{lstlisting}[language=Python]
# Filtering out the noise using the PSD
indices = PSD > 100
PSD_clean = PSD * indices # zeroing out all the PSD below 100
fhat = indices * fhat
f_filter = np.fft.ifft(fhat) # Inverse FFT
\end{lstlisting}

\begin{lstlisting}[language=Python]
# Now plotting the filtered data
fig, axs = plt.subplots(3,1)
plt.sca(axs[0])
plt.plot(t,f_filter,color='c',LineWidth='1.5',label='filtered signal')
plt.xlim(t[0],t[-1])
plt.legend()

plt.sca(axs[1])
plt.plot(freq[L],PSD_clean[L],color='c', LineWidth='1.5',label='PSD filtered')
plt.xlim(freq[L[0]], freq[L[-1]])
plt.legend()
plt.show()
\end{lstlisting}

\begin{lstlisting}
/usr/lib/python3.8/site-packages/numpy/core/_asarray.py:85: ComplexWarning: Casting complex values to real discards the imaginary part
  return array(a, dtype, copy=False, order=order)
/usr/lib/python3.8/site-packages/numpy/core/_asarray.py:85: ComplexWarning: Casting complex values to real discards the imaginary part
  return array(a, dtype, copy=False, order=order)
\end{lstlisting}

\begin{figure}
\centering
\includegraphics{FFT_files/FFT_6_1.png}
\caption{png}
\end{figure}

\begin{lstlisting}[language=Python]
\end{lstlisting}


\end{document}
